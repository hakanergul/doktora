\documentclass[a4paper, 10pt]{article}
\usepackage[familydefault,light]{Chivo}
\usepackage[T1]{fontenc}
\usepackage{mathpazo}
\renewcommand{\baselinestretch}{1.2} 

\usepackage[utf8]{inputenc}
\usepackage[turkish]{babel}
\usepackage{fullpage}
\usepackage{parskip} 
\usepackage{amsmath}
\usepackage{amsthm}
\usepackage{amsfonts}
\usepackage{enumitem}

\setenumerate[0]{label=\roman*.}
\title{\textbf{Fourier Dönüşümü ve Özellikleri}}
\author{Hakan ERGÜL\\
Öğr. No: o182119002 \\
\textit{Analiz ve Fonksiyonlar Teorisi Bilim Dalı}\\
\textit{Matematik Anabilim Dalı}\\
\textbf{Giresun Üniversitesi}}
\date{16/11/2018}

\begin{document}

\maketitle

\section{Giriş}

\paragraph{}
Mühendislik ve bilimin birçok dalında uygulaması olan Fourier serileri, periyodik fonksiyonların sinüs ve kosinüs fonksiyonlarının (sonsuz) toplamı olarak yazılabilmesinde kullanılır. Fakat periyodik olmayan fonksiyonların bu şekilde periyodik fonksiyonların toplamı olarak ifade edilmesinde Fourier serileri yetersiz kalmaktadır. İşte burada Fourier dönüşümü kullanılmaktadır. Öncelikle Fourier dönüşümünün tanımını ve temel özelliklerini ifade edeceğiz.

\section{Tanımı}
\paragraph{}
Tanımımızı gerçel sayılar kümesinde tanımlı Lebesgue integrallenebilir fonksiyonlar uzayı $L^1(\mathbb{R})$ üzerinde vereceğiz.
\paragraph{}
Herhangi bir $f \in L^1(\mathbb{R})$ fonksiyonunun $L^1$ normu 
$$ \left \| f  \right \|_{ L^1(\mathbb{R}) } = \int_{-\infty }^{\infty }\left | f(x) \right | dx $$
şeklinde tanımlanır. $ \left \| f  \right \|_{ L^1(\mathbb{R})}$  ifadesi yerine sadelik açısından $\left \| f  \right \|_1$ gösterimini tercih edeceğiz. Şimdi Fourier dönüşümü tanımını verelim:\\

\textbf{Tanım.} İşte herhangi bir $f \in L^1(\mathbb{R})$ fonksiyonunun Fourier dönüşümü $t\in \mathbb{R}$ olmak üzere,
\[  \widehat{f}(t)=\int_{-\infty }^{\infty }f(x)e^{-itx}dx  \]
biçiminde tanımlanır.

\section{Özellikleri}

\paragraph{} Fourier dönüşümünün özelliklerini aşağıdaki teoremle verelim:

\textbf{Teorem 1.} Keyfi $f, g \in L^1(\mathbb{R})$ ve $t\in \mathbb{R}$ olmak üzere aşağıdaki özellikler sağlanır:

\begin{enumerate}
\item Herhangi $\alpha \in \mathbb{C}$ için $\widehat{(\alpha f+g)}(t)$ = $\alpha \widehat{f}(t)$ + $\widehat{g}(t)$ dir.
\item  $\overline{f}$, $f$ fonksiyonunun karmaşık eşleniğini göstermek üzere,
\[ \widehat{\overline{f}}(t) = \overline{\widehat{f}}(-t) \]
olur.
\item  Bir $a \in \mathbb{R}$  sayısı alalım. Her $x\in \mathbb{R}$  için $g(x)=f(x+a)$ ise $g\in L^1(\mathbb{R})$ ve her $t\in \mathbb{R}$ için 
\[ \widehat{g}(t) = e^{iat}\widehat{f}(t)\]
dir.
\item Bir $b\in \mathbb{R}$ sabit sayısı alalım. Her $x\in \mathbb{R}$ için $h(x)=f(x)e^{bx}$ ise $h\in L^1(\mathbb{R})$ ve her $t\in \mathbb{R}$ için
\[ \widehat{h}(t) = \widehat{f}(t-b) \]
dir.
\item $c$ sıfırdan farklı bir gerçel sayı ve her $x \in \mathbb{R}$ için $j(x)=f(cx)$ ise $j\in L^1(\mathbb{R})$ ve her $t\in \mathbb{R}$ için
\[ \widehat{j}(t)=\frac{\widehat{f}(t/c)}{\left |  c \right |} \] dir.

\end{enumerate}

\textit{\textbf{İspat.}} 
\begin{enumerate}
	\item Herhangi $\alpha \in \mathbb{C}$ için 
	\begin{eqnarray*}
	\widehat{(\alpha f+g)}(t) &=& \int_{-\infty }^{\infty } (\alpha f + g)(x) e^{-itx}dx \\
	&=& \int_{-\infty }^{\infty } \left[ \alpha f(x) + g(x)\right]  e^{-itx}dx \\
	&=& \alpha \int_{-\infty }^{\infty } f(x) e^{-itx}dx + \int_{-\infty }^{\infty } g(x) e^{-itx}dx \\
	&=& \alpha \widehat{f}(t) + \widehat{g}(t) 	
	\end{eqnarray*}
dir.
	\item  Bir $f$ fonksiyonunun karmaşık eşleniğinin Fourier dönüşümü, o fonksiyonun Fourier dönüşümünün karmaşık eşleniğine eşittir:
		\begin{eqnarray*}
		\widehat{\overline{f}}(t)  &=&  \int_{-\infty }^{\infty } \overline{f}(x) e^{-itx}dx  \\ &=&  \int_{-\infty }^{\infty } \overline{f}(x) \overline{e^{itx}}dx\\ &=& \int_{-\infty }^{\infty } \overline{f(x)e^{itx}} dx \\ &=& \overline{\int_{-\infty }^{\infty } f(x)e^{itx} dx} \\ &=& \overline{\widehat{f}(-t)}  
		\end{eqnarray*}
	elde edilir.
	
	\item İntegralde değişken değiştirme yapacağız, yani $y=x+a$ dersek
		\begin{eqnarray*}
		\widehat{g}(t)  &=&  \int_{-\infty }^{\infty } g(x) e^{-itx}dx  \\ &=&  \int_{-\infty }^{\infty } {f}(x+a) e^{-itx}dx\\ &=& \int_{-\infty }^{\infty } f(y)e^{-it(y-a)} dy \\ &=& e^{iat}\int_{-\infty }^{\infty } f(y)e^{-ity} dy \\ &=& e^{iat}\widehat{f}(t)  
		\end{eqnarray*}
		bulunur.
		
	\item $h(x)=f(x)e^{ibx}$ verilmiş. O halde
		\begin{eqnarray*}
		\widehat{h}(t)  &=&  \int_{-\infty }^{\infty } h(x) e^{-itx}dx  \\ &=&  \int_{-\infty }^{\infty } {f}(x)e^{ibx} e^{-itx}dx\\ &=& \int_{-\infty }^{\infty } f(x)e^{-i(t-b)x} dx \\ &=& \widehat{f}(t-b)  
		\end{eqnarray*}
		olur.

	\item Bir $c$ sıfırdan farklı gerçel sayısı için $j(x)=f(cx)$ verilmiş. Buradan
		\begin{eqnarray*}
		\widehat{j}(t)  &=&  \int_{-\infty }^{\infty } j(x) e^{-itx}dx  \\ 
		&=&  \int_{-\infty }^{\infty } f(cx) e^{-itx}dx
		\end{eqnarray*}
		olup $u=cx$ değişken değiştirmesi yapılırsa
		\begin{eqnarray*}
		&=& \frac{1}{c} \int_{c (-\infty) }^{c \infty } f(u)e^{-it\frac{u}{c}} du \\   
		\end{eqnarray*}
		olur. Fakat burada integralin sınırları $c$ nin işaretine göre değişir. Eğer $c>0$ ise
		\begin{eqnarray*}
		&=& \frac{1}{c} \int_{c (-\infty) }^{c \infty } f(u)e^{-it\frac{u}{c}} du \\   
		&=& \frac{1}{c} \int_{-\infty }^{\infty } f(u)e^{-it\frac{u}{c}} du \\  
		&=& \frac{1}{c} \widehat{f}(\frac{t}{c}) 
		\end{eqnarray*}
		tersine $c<0$ ise
		\begin{eqnarray*}
		&=& \frac{1}{c} \int_{c (-\infty) }^{c \infty } f(u)e^{-it\frac{u}{c}} du \\   
		&=& \frac{1}{c} \int_{\infty }^{-\infty } f(u)e^{-it\frac{u}{c}} du \\  
		&=& -\frac{1}{c} \widehat{f}(\frac{t}{c}) 
		\end{eqnarray*}
		elde edilir. Sonuç olarak
		$$ \widehat{j}(t) = \frac{1}{\left | c \right |} \widehat{f}(\frac{t}{c}) $$
		bulunur.\qed
\end{enumerate}

Yukarıdaki teoremde Fourier dönüşümünün bazı özelliklerini verdik. Aşağıdaki teoremle bir $f\in L^1(\mathbb{R})$ fonksiyonunun tek olması halinde, fonksiyonun Fourier dönüşümü olan $\widehat{f}$ nin de tek olacağını göstereceğiz. Benzer durum çift olmaları halinde de geçerlidir.

\textbf{Teorem 2.} Keyfi $f \in L^1(\mathbb{R})$ tek ise  $\widehat{f}$ de tektir.\\
\textit{\textbf{İspat.}} 
Herhangi bir $f \in L^1(\mathbb{R})$ tek fonksiyonu alalım. Buradan	$\widehat{f}(t) =  \int_{-\infty }^{\infty } f(x) e^{-itx}dx$ olduğunu biliyoruz. İntegralde $y=-x$ değişken değiştirmesi yaparsak  
	\begin{eqnarray*} 
		\widehat{f}(t) &=&  \int_{-\infty }^{\infty } f(x) e^{-itx}dx\\
		&=& - \int_{-\infty }^{ \infty } f(-y)e^{-it(-y)} dy \\   
		&=& -\widehat{f}(-t) 
	\end{eqnarray*}
Benzer olarak $f$ çift olduğunda $\widehat{f}$ nin de çift olacağı gösterilebilir.\qed

\textbf{Teorem 3.} Herhangi bir $f\in L^1(\mathbb{R})$ ve bu fonksiyonun Fourier dönüşümü $\widehat{f}$ verilsin. O halde

$$ \left| \widehat{f}(t) \right| \leq \int_{-\infty}^{\infty} \left| f(x) \right| dx = \left \| f \right \|_{1}  $$
dir. \\
\textit{\textbf{İspat.}} Biliyoruz ki bir $f \in L^1(\mathbb{R})$ fonksiyonu için $\widehat{f}(t) =  \int_{-\infty }^{\infty } f(x) e^{-itx}dx$ dir. Her iki tarafın mutlak değerini alırsak
	\begin{eqnarray*} 
		\left| \widehat{f}(t) \right| &=& \left| \int_{-\infty }^{\infty } f(x) e^{-itx}dx \right| \\
		&\leq &  \int_{-\infty }^{ \infty } \left| f(x)e^{-it(x)} \right| dx \\   
		&=& \int_{- \infty}^{\infty} \left| f(x) \right| \underbrace{\left| e^{-itx} \right| }_{1} dx \\
		&=& \int_{- \infty}^{\infty} \left| f(x) \right| dx \\
		&=& \left \| f \right \|_{1}
	\end{eqnarray*}
sonucunu elde ederiz. \qed

\section{Girişim}
\paragraph{}
Bu başlıkta girişimin tanımını verip Fourier dönüşümü ile ilgili bir özelliğini vereceğiz.

\textbf{Tanım} $f,g \in L^1(\mathbb{R})$ olmak üzere $f$ ve $g$ fonksiyonlarının girişimi
\begin{eqnarray*} 
 \left( f\ast g \right) (x) &=& \int_{- \infty}^{\infty} f(y) g(x-y) dy
\end{eqnarray*}
şeklinde tanımlanır. Girişimin, değişmelilik özelliği vardır. Gerçekten tanımda verilen integralde $t=x-y$ değişken değiştirmesi yapılırsa $g\ast f$ elde edilir.
\paragraph{}
$f,g \in L^1(\mathbb{R})$ olarak aldığımızda bu iki fonksiyonun girişimlerinin Fourier dönüşümü, Fourier dönüşümlerinin çarpımına eşittir. Yani Fourier dönüşümü sayesinde girişim işlemi uygulanmakta zorluk çekilen durumlarda, fonksiyonların Fourier dönüşümlerini çarpıp daha sonra ters Fourier dönüşümü uygulamak işimizi kolaylaştırabilir. Şimdi bu özelliği ifade eden teoremi verelim:

\textbf{Teorem 4.} Keyfi $f,g \in L^1(\mathbb{R})$ fonksiyonları için $\left( f\ast g \right) \in L^1(\mathbb{R})$ dir. Ayrıca 
$$ \widehat{\left( f\ast g \right)}(t) = \widehat{f}(t) + \widehat{g}(t)  $$
dir. \\
\textit{\textbf{İspat.}} 
Öncelikle $\left( f\ast g \right) \in L^1(\mathbb{R})$ olduğunu gösterelim. $f, g \in L^1(\mathbb{R})$ olmak üzere
	\begin{eqnarray*} 
		\int_{-\infty}^{\infty} \left| \left( f\ast g \right)(x) \right|dx &=&  \int_{-\infty}^{\infty} \left| \int_{-\infty}^{\infty} f(x-u)g(u)du \right|dx \\
		&\leq &  \int_{-\infty}^{\infty} \int_{-\infty}^{\infty} \left| f(x-u)g(u) \right| du dx 
	\end{eqnarray*}
olup çift katlı integral mutlak yakınsak olduğundan Fubini Teoremi'nden integral alma sırasını değiştirebiliriz. Yani
	\begin{eqnarray*}
	&=&  \int_{-\infty}^{\infty} \int_{-\infty}^{\infty} \left| f(x-u)g(u) \right| dx du \\
	&=&  \int_{-\infty}^{\infty} g(u) \underbrace{\left( \int_{-\infty}^{\infty} \left| f(x-u) \right| dx \right) }_{\left \| f \right \|_{1}} du \\
	&=&  \int_{-\infty}^{\infty} g(u)\left \| f \right \|_{1} du \\
	&=& \left \| f \right \|_{1} \underbrace{\int_{-\infty}^{\infty} g(u) du}_{{\left \| g \right \|}_{1}} \\
	&=& \left \| f \right \|_{1}\left \| g \right \|_{1}
	\end{eqnarray*}
elde edilir. Eşitsizliğin sol tarafındaki integralin sonlu olmasından dolayı $f\ast g \in L^1(\mathbb{R})$ dir. 
Şimdi $ \widehat{\left( f\ast g \right)}(t) = \widehat{f}(t) + \widehat{g}(t)  $ olduğunu gösterelim:
	\begin{eqnarray*}
	\widehat{\left( f\ast g \right)}(t) &=& \int_{-\infty}^{\infty}  \left( f\ast g \right)(x)  e^{-itx} dx\\
	 &=&  \int_{-\infty}^{\infty} \left[ \int_{-\infty}^{\infty}  f(x-u)g(u) du \right] e^{-itx} dx 
	\end{eqnarray*}
yine integral mutlak yakınsak olduğu için integral alma sırasını değiştirirsek
	\begin{eqnarray*}
	 &=& \int_{-\infty}^{\infty} \left[ \int_{-\infty}^{\infty} f(x-u)g(u) e^{-itx} dx \right] du \\
	 &=&  \int_{-\infty}^{\infty} g(u) e^{-itu} \left[ \int_{-\infty}^{\infty}  f(x-u)e^{-itx} e^{itu} dx \right] du \\
	 &=&  \int_{-\infty}^{\infty} g(u) e^{-itu} \underbrace{\left[ \int_{-\infty}^{\infty}  f(x-u)e^{-it(x-u)} dx \right]}_{\widehat{f}(t)} du \\
	 &=&  \widehat{f}(t) \underbrace{\int_{-\infty}^{\infty} g(u) e^{-itu} du}_{\widehat{g}(t)} \\
	 &=&  \widehat{f}(t)\widehat{g}(t)	 
	\end{eqnarray*}
sonucunu buluruz.\qed

\begin{thebibliography}{9}
\bibitem{MONT}
Hugh L. Montgomery,
\textit{Early Fourier Analysis},
AMS,
390 sf.,
2014

\bibitem{KATZ}
Yitzhak Katznelson,
\textit{An Introduction to Harmonic Analysis},
Cambridge University Press,
287 sf.,
2002

\end{thebibliography}


\end{document}